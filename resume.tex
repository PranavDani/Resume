%-------------------------
% Resume in Latex
% Author: Jake Gutierrez
% Based off of: https://github.com/sb2nov/resume
% License: MIT
%------------------------

%------------------------
% Original Overleaf template: https://www.overleaf.com/latex/templates/jakes-resume-anonymous/cstpnrbkhndn
%------------------------

\documentclass[letterpaper,11pt]{article}

\usepackage{latexsym}
\usepackage[empty]{fullpage}
\usepackage{titlesec}
\usepackage{marvosym}
\usepackage[usenames,dvipsnames]{color}
\usepackage{verbatim}
\usepackage{enumitem}
\usepackage[hidelinks]{hyperref}
\usepackage{fancyhdr}
\usepackage[english]{babel}
\usepackage{tabularx}
\usepackage{fontawesome5}
\usepackage{multicol}
\setlength{\multicolsep}{-3.0pt}
\setlength{\columnsep}{-1pt}
\input{glyphtounicode}

\usepackage{xcolor}
\hypersetup{colorlinks,urlcolor=blue}


%----------FONT OPTIONS----------
% sans-serif
% \usepackage[sfdefault]{FiraSans}1
% \usepackage[sfdefault]{roboto}
% \usepackage[sfdefault]{noto-sans}
\usepackage[default]{sourcesanspro}

% serif
% \usepackage{CormorantGaramond}
% \usepackage{charter}
% \usepackage{times}
% \usepackage{helvet}


\pagestyle{fancy}
\fancyhf{} % clear all header and footer fields
\fancyfoot{}
\renewcommand{\headrulewidth}{0pt}
\renewcommand{\footrulewidth}{0pt}

% Adjust margins
\addtolength{\oddsidemargin}{-0.6in}
\addtolength{\evensidemargin}{-0.5in}
\addtolength{\textwidth}{1.19in}
\addtolength{\topmargin}{-.7in}
\addtolength{\textheight}{1.4in}
\setlength{\footskip}{4.08003pt}

\urlstyle{same}

\raggedbottom
\raggedright
\setlength{\tabcolsep}{0in}

% Sections formatting
\titleformat{\section}{
  \vspace{-5pt}\scshape\raggedright\large\bfseries
}{}{0em}{}[\color{black}\titlerule \vspace{-5pt}]

% Ensure that the generated PDF is machine readable/ATS parsable.
\pdfgentounicode=1

%-------------------------
% Custom commands
\newcommand{\resumeItem}[1]{
  \item\small{
    {#1 \vspace{-2pt}}
  }
}

\newcommand{\classesList}[4]{
    \item\small{
        {#1 #2 #3 #4 \vspace{-2pt}}
  }
}

\newcommand{\resumeSubheading}[4]{
  \vspace{-2pt}\item
    \begin{tabular*}{1.0\textwidth}[t]{l@{\extracolsep{\fill}}r}
      \vspace{-2pt}\textbf{#1} & \small #2 \\
      \textit{\small#3} & \textit{\small #4} \\
    \end{tabular*}\vspace{-7pt}
}

\newcommand{\resumeSubSubheading}[2]{
    \item
    \begin{tabular*}{0.97\textwidth}{l@{\extracolsep{\fill}}r}
      \textit{\small#1} & \textit{\small #2} \\
    \end{tabular*}\vspace{-7pt}
}

\newcommand{\resumeProjectHeading}[2]{
    \item
    \begin{tabular*}{1.001\textwidth}{l@{\extracolsep{\fill}}r}
      \small#1 & \small #2\\
    \end{tabular*}\vspace{-7pt}
}

\newcommand{\resumeSubItem}[1]{\resumeItem{#1}\vspace{-4pt}}

\renewcommand\labelitemi{$\vcenter{\hbox{\tiny$\bullet$}}$}
\renewcommand\labelitemii{$\vcenter{\hbox{\tiny$\bullet$}}$}

\newcommand{\resumeSubHeadingListStart}{\begin{itemize}[leftmargin=0.0in, label={}]}
\newcommand{\resumeSubHeadingListEnd}{\end{itemize}}
\newcommand{\resumeItemListStart}{\begin{itemize}}
\newcommand{\resumeItemListEnd}{\end{itemize}\vspace{-5pt}}

%------------------z-------------------------
%%%%%%  RESUME STARTS HERE  %%%%%%%%%%%%%%%%%%%%%%%%%%%%


\begin{document}


%----------HEADING----------
% \begin{tabular*}{\textwidth}{l@{\extracolsep{\fill}}r}
%   \textbf{\href{http://sourabhbajaj.com/}{\Large Sourabh Bajaj}} & Email : \href{mailto:sourabh@sourabhbajaj.com}{sourabh@sourabhbajaj.com}\\
%   \href{http://sourabhbajaj.com/}{http://www.sourabhbajaj.com} & Mobile : +1-123-456-7890 \\
% \end{tabular*}

\begin{center}
  {\huge Pranav Dani} \\[4pt]
  \small
  % Mobile on its own line with icon name
  \textbf{Mobile:} \href{tel:+19344529426}{+1 (934) 451-9426}
  \quad   \faEnvelope \href{mailto:contact@pranavdani.com}{\raisebox{-0.2\height}\    contact@pranavdani.com}\\
  \faGithub \href{https://github.com/PranavDani}{\raisebox{-0.2\height}\  github.com/PranavDani} \quad
  \faLinkedin \href{https://linkedin.com/in/pranav-dani}{\raisebox{-0.2\height}\  linkedin.com/in/pranav-dani} \quad
  \faGlobe \href{https://pranavdani.com}{\raisebox{-0.1\height}\ pranavdani.com}
  \vspace{-10pt}
\end{center}


%-----------EDUCATION-----------
\section{Education}
\resumeSubHeadingListStart
\resumeSubheading
{SUNY -- Stony Brook University}{Aug 2023 -- May 2025}
{Master of Science, Computer Science}{New York, USA}
\resumeItemListStart
\resumeItem{Courses: Computer Architecture, OS, Distributed Systems, System Security, Theory of Databases, Analaysis of Algorithms}
\vspace{-2pt}
\resumeItem{Teaching Assistant: CSE 316: Fundamentals of Software Development}
\resumeItemListEnd
\vspace{-3pt}

\resumeSubheading
{University of Mumbai -- Thadomal Shahani Engineering College}{Aug 2019 -- May 2023}
{Bachelor of Engineering, Information Technology}{Mumbai, India}
\resumeSubHeadingListEnd
% \vspace{-13pt}

%-----------SKILLS-----------

\section{Technical Skills}
\begin{itemize}[leftmargin=0.2in, label={}]
  \item \textbf{Languages and Databases:} C, C++, Java, Python, Go, System Verilog; PostgreSQL, MySQL, MongoDB, Firebase\\
        \textbf{Tools/Web/CI:} Unix/Linux, Docker, QEMU, GTKWave, Kubernetes,  React.js, Node.js, Flask, HTML/CSS/JS, AWS-EC2, S3
\end{itemize}
\vspace{-16pt}


%-----------EXPERIENCE-----------
\section{Work Experience}
\resumeSubHeadingListStart
% \vspace{-2pt}

\resumeSubheading
{Graduate Research Assistant: GPU and CPU Profiling}{May 2024 -- Present}
{Stony Brook University}{New York, US}
\resumeItemListStart
\resumeItem{
  Engineered a CPU Energy Flamegraph tool using Linux perf\_events, eBPF and \href{https://powerapi.org/}{PowerAPI} to trace CPU call chains and monitor power consumption per cgroup, enhancing energy efficiency analysis for developers.
}
% Developed an Energy Flamegraph tool using Linux perf\_events and modifying PowerAPI to trace CPU call chains and monitor power consumption per cgroup. This tool generates both CPU and energy flamegraphs from the same execution trace.
\resumeItem{
  Crafted a GPU Energy Flamegraph tool using CUPTI and NVML to monitor GPU power consumption per kernel, enhancing GPU power usage insights for optimization.
}
% Developed a tool to collect GPU API and kernel launch traces, along with power consumption data, using CUPTI and NVML, which improved the generation of Energy Flamegraphs for better analysis of GPU energy usage.
\resumeItemListEnd
\vspace{-3pt}

\resumeSubheading
{Software Intern}{Jun 2021 -- Aug 2021}
{Suven Consultants}{Mumbai, India}
\resumeItemListStart
\resumeItem{Devised a Home Inventory and Loan Management tool using Java and SQLite3; gained 150+ active users in the first month.}
\resumeItem{Implemented an advanced Printable interface with Java AWT and the Graphics Library to generate professional PDF reports in under 2 seconds per report—boosting document accessibility by 75\% and processing over 150 reports weekly.}
\resumeItemListEnd
\vspace{-6pt}
\resumeSubHeadingListEnd
% \vspace{-22pt}


%-----------PROJECTS-----------
\section{Projects}
\vspace{-4pt}
\resumeSubHeadingListStart

\resumeProjectHeading
{\textbf{Computer Architecture} -- \textbf{\href{https://riscv.org/wp-content/uploads/2017/05/riscv-spec-v2.2.pdf}{RISC-V} Processor} $|$ \emph{System Verilog, GTKWave, C}}{Jun 2024 -- Oct 2024}
\resumeItemListStart
\resumeItem{Designed a synthesizable multi-cycle in-order RISC-V (RV64IM) processor which communicates with memory over \href{https://developer.arm.com/documentation/ihi0022/latest}{\textbf{AXI4 protocol}}}
\resumeItem{Implemented an ALU to execute instructions, and interact with reg file, supporting pipeline stalls syscalls.}
\resumeItem{Integrated branch prediction, 2 set-associative L1 caches, load/stores and ECALL instructions with pipeline flush.}
\resumeItemListEnd
\vspace{-17pt}

\resumeProjectHeading
{\textbf{Kernel Programming} -- \textbf{File Systems} $|$ \emph{C, QEMU}}{Mar 2024 -- May 2024}
\resumeItemListStart
\resumeItem{
  Constructed an asynchronous journaling protocol in xv6, reducing disk write() latency by up to 94\%.
}
% Developed an asynchronous disk logging protocol for \href{https://pdos.csail.mit.edu/6.828/2012/xv6.html}{\textbf{xv6}}, optimizing write efficiency by caching writes and writing to disk when writing, achieving a 94\% reduction in write latency.
\resumeItem{
  Added small file support with file type conversion, optimizing disk space utilization and reducing disk I/O by 95\% for files \textless{} 52B.
}
\resumeItemListEnd
\vspace{-17pt}

\resumeProjectHeading
{\textbf{\normalsize{Distributed Systems -- Key-Value Store with \href{https://raft.github.io/raft.pdf}{Raft} Consensus}} $|$ \emph{C++}}{Aug 2023 -- Dec 2023}
\resumeItemListStart
\resumeItem{Architected a persistent k-v store using Raft for leader election and data replication. Added snapshotting for quick recovery.}
\resumeItem{Executed sharding with consistent hashing for efficient data distribution and automated partition rebalancing.}
\resumeItem{Formulated a versioned key-value store that supports cross-shard transactions using 2-Phase Locking and 2-Phase Commit with Optimistic Concurrency Control.}
% Created a persistent linearizable key-value store using the Raft consensus algorithm for leader election and data replication in an asynchronous environment. Implemented log compaction with Snapshots.
% Engineered sharding using consistent hashing, ensuring efficient data distribution across nodes. Automated partition rebalancing during node joins and departures.
\resumeItemListEnd
\vspace{-17pt}

\resumeProjectHeading
{\textbf{\normalsize{Unix Systems Programming Projects}} $|$ C, C++, Python, Perl, Bash, QEMU}{Jan 2023 -- Present}
\resumeItemListStart
\resumeItem{\textbf{KV Store}: A multithreaded key-value store with distributed transactions, supporting multiple clients and persistence.}
\resumeItem{
  \textbf{\textit{ftruncate()}}: A Unix system call for adjusting the file size—either increasing or decreasing it.
}
\resumeItem{\textbf{Locks:} Wrote an RCU-based lock supporting concurrent readers and a single writer, ensuring atomic access to shared resources.}
\resumeItem{\textbf{GPU Flamegraph}: A tool to visualize GPU (CUDA) kernel execution and power consumption through \textit{NVML} and \textit{nsys}.}
\resumeItemListEnd
\vspace{-17pt}

\resumeProjectHeading
{\textbf{\normalsize{BackGen - GoLang Backend Generator}} $|$ \emph{ICT4SD $|$ \href{https://link.springer.com/chapter/10.1007/978-981-99-6568-7_34}{Springer}}}{Jan 2023 -- Aug 2023}
\resumeItemListStart
\resumeItem{Engineered a GoLang backend generator that generates server code based on REST API spec, reducing dev time by 50\%.}
\resumeItem{Validated on a Todo application, the tool generates nearly 48\% of the code, significantly streamlining web app development.}
\resumeItemListEnd
\vspace{-17pt}

\resumeProjectHeading
{\textbf{\normalsize{Web Projects}} $|$ \emph{React.js, Node.js, Flask, PostgreSQL, Heroku, HTML, CSS, JS, AWS-EC2, S3, SQL}}{Apr 2020 -- Present}
\resumeItemListStart
\resumeItem{{\textbf{\href{https://github.com/PranavDani/Expense-Tracker}{\textcolor{black}{Expense Tracker}}} -- Web app for tracking expenses with bulk expense creation and file export, attracting 100+ users in a month.}
}
\resumeItem{\textbf{\href{https://github.com/raunaqpahwa/cse564}{\textcolor{black}{NYC Housing}}} -- A d3 based web app for visualizing NYC housing data, enabling users to filter and analyze various parameters.}
\resumeItem{\textbf{\href{https://github.com/PranavDani/short-terms}{\textcolor{black}{Short-Terms}}} -- Chrome extension for summarizing web pages using NLP and Spcay (before GPT).}
\resumeItemListEnd

% \resumeProjectHeading
% {\textbf{\normalsize{Expense Tracker}} $|$ \emph{Flask, PostgreSQL, Heroku, HTML, CSS, JS} $|$ \href{https://github.com/PranavDani/Expense-Tracker}{GitHub}}{Apr 2021 -- Jun 2021}
% \resumeItemListStart
% \resumeItem{Built an expense tracker web app with bulk expense creation and CSV/Excel export, attracting 100+ users within first month.
% }
% \resumeItem{Enabled personalized budget creation across multiple categories, enhancing flexibility in expense tracking.}
% \resumeItemListEnd

\resumeSubHeadingListEnd
\vspace{-12pt}

%-----------EXTRACURRICULAR ACTIVITY-----------
\section{Extracurricular Activity}
\vspace{-4pt}
\resumeSubHeadingListStart
\resumeProjectHeading
{\textbf{\normalsize{Our Tech Community (OTC)}} $|$ \normalfont\href{https://ourtech.community}{ourtech.community} $|$ \emph{Admin}}{May 2022 -- Present}
\resumeItemListStart
\resumeItem{Hosted 400+ hours of weekly \href{https://catchup.ourtech.community}{OTC CatchUp} technical discussions, organized two in-person \href{https://meetup.ourtech.community}{MeetUp} events with 70+ attendees.}
\resumeItemListEnd
% \vspace{-18pt}
\resumeSubHeadingListEnd
% \vspace{-8pt}

\end{document}